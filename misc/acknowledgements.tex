\documentclass{article}
\usepackage{hyperref}
\usepackage{graphicx}
\usepackage[utf8]{inputenc}
\usepackage{geometry}
\geometry{a4paper, margin=1in}

\title{Project Acknowledgements}
\author{}
\date{\today}

\begin{document}
\maketitle

\section*{Introduction}
This document outlines the sources of inspiration, tools, and collaborative efforts involved in the development of the UniMind web application.

\section*{Initial Structure and AI Collaboration}
The foundational structure and template for the web application were initially generated using \href{https://bolt.new/}{bolt.new}, leveraging \textbf{Claude 3.7 Sonnet}. This involved multiple collaborative prompts and debugging sessions, also heavily utilizing AI assistance. The development process followed the features outlined in documents within the \texttt{misc} directory, implementing them incrementally.

\section*{Feature Inspirations}
\begin{itemize}
    \item \textbf{Chat Interface:} The design goal was to emulate the simplicity and functionality of the \href{https://chatgpt.com/}{ChatGPT} interface.
    \item \textbf{UI Underglow Effect:} The subtle underglow effect present on various pages was inspired by the implementation seen on the \href{https://bolt.new/}{bolt.new} website.
    \item \textbf{Logo:} The logo's aesthetic drew inspiration from the branding noted on the \href{https://allenai.org/}{Allen Institute for AI (AI2)} website, which was observed during parallel work on the separate iOS version of the application. The final logo image was generated using GPT-4o's image generation capabilities.
    \item \textbf{Companion Mode:} The initial concept for the interactive "Companion Mode" orb was inspired by similar elements on the \href{https://elevenlabs.io/app/conversational-ai/agents}{Eleven Labs} website. This concept evolved, taking further inspiration from space, particularly the visual transformation process of a star into a neutron star, aiming to create an engaging and visually interesting user experience.
    \item \textbf{Mood Tracker:} The primary inspiration for the Mood Tracker feature came from the user interface and functionality observed in the \href{https://www.youper.ai/}{Youper} iOS application.
\end{itemize}

\section*{Development Process and Testing}
While AI played a significant role in generating initial code and assisting with debugging, substantial effort was dedicated to testing and refinement. This included:
\begin{itemize}
    \item Writing unit tests where applicable.
    \item Extensive use of console logging to trace behaviour and debug features.
    \item Utilising external resources like Perplexity for research on common debugging techniques (e.g., addressing path case-sensitivity differences between Linux, Windows, and macOS).
\end{itemize}

\section*{Connection to iOS Version}
It's worth noting that some features, particularly the Mood Tracker, were initially conceptualised or built in a similar form within the separate native iOS project, available at \href{https://github.com/ryantigi254/Mental-Health-Chatbot-App}{ryantigi254/Mental-Health-Chatbot-App}, before being adapted for this web application.

\section*{Further Details on AI Collaboration}
For a more detailed breakdown of how AI was specifically used throughout the project development lifecycle, please refer to the \texttt{misc/ai\_collaboration\_report.tex} document.

\end{document} 